\documentclass[oneside,openright,9pt,a4paper,headings]{jsarticle}

\usepackage[dvips]{graphicx}
\usepackage[dvips]{graphicx,color}
\usepackage{amsmath}
\usepackage{makeidx}
\usepackage{url}
\usepackage{multirow}
\usepackage{multicol}
\usepackage{booktabs}
\usepackage{txfonts}
\usepackage[scaled]{helvet}
\usepackage{colortbl}
\usepackage{enumerate}
\usepackage{textcomp}
\usepackage{ascmac}
\usepackage{fancybox}
\usepackage[utf8]{inputenc}
\usepackage[version=3]{mhchem}

\begin{document}
\section{C - Dice and Coin}

% https://atcoder.jp/contests/abc126/tasks/abc126_c

サイコロを投げたとき,$i$が出たとする.

$i$ が出る確率は $ \displaystyle \frac{1}{n} $ である.

$ k \leq i \leq n $ のとき,コインを振る必要はないので,サイコロで $i$ が出たときに勝つ条件付き確率は $1$ である.

$ 1 \leq i < k $ のとき,勝つためにコインを振って表を出さなければならない回数を $ t_{i} $ とする.
$ t_{i} $回コインを振り続け,表が出続けたとすると,$ t_{i} $回コインを振ったときの得点 $ \mathrm{Point} (i) $ について,
\begin{align}
  \mathrm{Point} (i) &= i \cdot 2^{ t_{i} } \label{eq:point_1} \\
  \mathrm{Point} (i - 1) & < k \leq \mathrm{Point} (i) \label{eq:point_2}
\end{align}
が成り立つ.

\eqref{eq:point_1}, \eqref{eq:point_2} より,
\begin{align}
  i \cdot 2^{ t_{i} - 1 } & < k \leq i \cdot 2^{ t_{i} } \notag \\
  2^{ t_{i} - 1 } & < \frac{k}{i} \leq 2^{ t_{i} } \notag \\
  t_{i} - 1 & < \log_{2} \left( \frac{k}{i} \right) \leq t \notag
\end{align}
が成り立つ.

ここで,天井関数 $ \lceil x \rceil $ を用いると,
\begin{align}
  t_{i}
    &= \Biggl\lceil \log_{2} \left( \frac{k}{i} \right) \Biggr\rceil \notag \\
    &= \lceil \log_{2} k - \log_{2} i \rceil
\end{align}
と表せる.

以上より,サイコロを投げて $i$ が出た場合(最初の得点が $i$ の場合)のときに勝つ条件付き確率 $ p_{i} $ は,
\begin{align}
  p_{i} &= %
  \begin{cases}
    \qquad 1 & \qquad \text{ ( $ k \leq i \leq n $ ) } \\[6pt]
    \qquad { \left( \frac{1}{2} \right) }^{ \lceil \log_{2} k - \log_{2} i \rceil } & \qquad \text{(上記以外)}
  \end{cases}
\end{align}
である.

求める確率 $p$ は,
\begin{align}
  p &= \sum_{i=1}^{n} \, \left( \frac{1}{n} p_{i} \right) \\
    &= \frac{1}{n} \, \left( \sum_{i=1}^{n} \, p_{i} \right)
\end{align}
であり,$ \displaystyle \sum_{i=1}^{n} \, p_{i} $ について,$ k \leq n $ のとき,
\begin{align}
  \sum_{i=1}^{n} \, p_{i} &= ( n - k ) + \sum_{i=1}^{k} \, p_{i}
\end{align}
であるから,
\begin{align}
  p &= %
    \begin{cases}
      \qquad \displaystyle \frac{1}{n} \, \left\{ ( n - k ) + \sum_{i=1}^{k} \, p_{i} \right\} & \qquad \text{ ( $ k \leq n $ ) } \\[12pt]
      \qquad \displaystyle \frac{1}{n} \, \left( \sum_{i=1}^{n} \, p_{i} \right) & \qquad \text{(上記以外)}
    \end{cases}
\end{align}
である.

\end{document}
